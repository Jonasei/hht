\section{Systematic Literary Review}
\label{SLR}

This is the full version of the systematic literary research. It describes how and where the sources used in this project can be found, as well as the rationale for selecting the sources that are used. The reason for including a systematic literary review is to help the reader understand why certain articles are included, while others are not. 



\subsection{Sources}




\begin{table}
\centering
\begin{tabular}{|l|}
\hline
\textbf{Source}\\ \hline
IEEE Xplore\\ \hline
SpringerLink\\ \hline
Scopus\\ \hline
\end{tabular}
\caption{Databases used for literary search}
\label{sources}
\end{table}

Table \ref{sources} show the databases used for finding the articles cited in this project. The reasons for choosing these databases is that they are very popular and reliable, especially in the fields related to this project.


\subsection{Search Terms}


\begin{table}
\centering
\begin{tabular}{|l|l|l|l|l|}
\hline
 					& \textbf{Group 1} 			& \textbf{Group 2} 	& \textbf{Group 3} 	& \textbf{Group 4}	\\ \hline
\textbf{Term 1} 	& Hilbert Huang Transform 	& Forecast			& Time series		& Smart Grid		\\ \hline
\textbf{Term 2} 	& HHT 						& Prediction		& 					& Power Grid		\\ \hline
\textbf{Term 3} 	& 							& Forecasting		& 					& Electrical Grid	\\ \hline
\end{tabular}
\caption{Phrases used when searching}
\label{keywords}
\end{table}

When searching for literature related to this project, the search phrases in Table \ref{keywords} was used. They were put together in a search string in an attempt to get only relevant articles. The phrases was combined like this (T being the term, and G being the group):

\[([G1,T1]\vee[G1,T2])\land  ([G2,T1]\vee[G2,T2]\vee[G2,T3])\land ([G3,T1]) \land ([G4,T1]\vee[G4,T2]\vee[G4,T3])   \]

However, this resulted in only one hit in the databases. This is far from enough, so the search was repeated without Group 4. This gave better results, while still being relevant. 

\subsection{Research Questions}
\begin{table}
\centering
\begin{tabular}{|l|l|} \hline
RQ1	& Can Hilbert-Huang Transform (HHT) be used for load forecasting? \\ \hline
RQ2	& How can the HHT be use for prediction? \\ \hline
RQ3	& Does using HHT for prediction offer any improvements over other methods of prediction? \\ \hline
\end{tabular}
\caption{Research questions}
\label{researchquestions}
\end{table}

The research questions in Table \ref{researchquestions} are the questions that the report will try to answer. 

\subsection{Inclusion Criteria}
\begin{table}
\centering
\begin{tabular}{|l|l|} \hline
IC1 & The article's main concern is using HHT for prediction.\\ \hline
IC2 & The article describes how and why HHT was used\\ \hline
IC3 & The article compares HHT to other methods for prediction\\ \hline
IC4 & The article has empirical proof, or cites another article, when making statements\\ \hline
\end{tabular}
\caption{Inclusion criteria}
\label{inclusioncriteria}
\end{table}

The articles that got selected from the search results was selected using the inclusion criteria found in Table \ref{inclusioncriteria}. IC1 was of course the most important inclusion criteria. All articles selected had to, at least, fulfil that one, as well as one or more of the other criteria.


\subsection{Quality Assessment}

\begin{table}
\centering
\begin{tabular}{|l|l|} \hline
QC1 & The study is put into context of other studies\\ \hline
QC2 & The aim of the research is clearly stated\\ \hline
QC3 & Decisions made are clearly justified\\ \hline
QC4 & The results and procedures are thoroughly explained and analysed \\ \hline
QC5 & The article is structured and uses good language \\ \hline
\end{tabular}
\caption{Quality criteria}
\label{qualitycriteria}
\end{table}

The articles that got selected using the inclusion criteria described, are rated using the quality criteria from Table \ref{qualitycriteria}. Each article got a score for each quality criteria. The scores where summed up, and the ones with the highest scores were considered the best articles, and were given more weight when doing this project.




